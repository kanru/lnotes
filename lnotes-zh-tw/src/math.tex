\chapter{數學}

\begin{quotation}
今有上禾三秉,中禾二秉,下禾一秉,實三十九斗;上禾二秉,中禾三秉,下禾一秉,實三十四斗;上禾一秉,中禾二秉,下禾三秉,實二十六斗。問上、中、下禾實一秉各幾何?
\begin{gather*}
\begin{split}
    3x+2y+z &= 39 \\
    2x+3y+z &= 34 \\
    x+2y+3z &= 26 \\
\end{split}
\end{gather*}
\begin{flushright}
--- 《九章算術》
\end{flushright}
\end{quotation}

%\begin{quotation}
%鄴下諺曰:博士買驢,書券三紙,未有驢字。
%\begin{flushright}
%--- 顏之推《顏氏家訓‧勉學》
%\end{flushright}
%\end{quotation}

為了使用~\AmS-\LaTeX~提供的數學功能,我們首先需要在文檔的序言部分加載~\verb|amsmath|~宏包,其詳細用法可參閱《amsmath User's Guide》\citep{AMS_2002}。更全面的數學內容排版可參閱~George Grätzer\footnote{匈牙利裔,加拿大~Manitoba~大學數學系教授。}的《More Math into \LaTeX, 4th Edition》\citep{Gratzer_2007}。
%\begin{code}
%\usepackage[options]{amsmath}
%\end{code}

\section{數學模式}
\LaTeX~中的數學模式有兩種形式:inline~和~display。前者是指在正文插入行間數學公式,後者獨立排列,可以有或沒有編號。

行間公式用一對~\verb|$...$|~來輸入,獨立公式用~\verb|equation|~或~\verb|equation*|~環境來輸入,有~\verb|*|~的版本不生成公式編號。

前文提到~\verb|\fbox|~命令可以給文本內容加個方框,數學模式下也有類似的命令~\verb|\boxed|。

\begin{demo}
愛因斯坦的$E=mc^2$方程
\begin{equation} 
    E=mc^2 
\end{equation}
\[ E=mc^2 \]
\[ \boxed{E=mc^2} \]
\end{demo}

\section{基本原素}
\subsection{字母}

英文字母在數學模式下可以直接輸入,希臘字母則需要用\Fref{tab:greek}~中的命令輸入,注意大寫希臘字母的命令首字母也是大寫。

\begin{table}[htbp]
\caption{希臘字母}
\label{tab:greek}
\centering
\begin{tabular}{llllllll}
    \toprule
    $\alpha$      & \verb|\alpha|      & $\theta$    & \verb|\theta|    & 
        $o$         & \verb|o|        & $\tau$     & \verb|\tau| \\
    $\beta$       & \verb|\beta|       & $\vartheta$ & \verb|\vartheta| & 
        $\pi$       & \verb|\pi|      & $\upsilon$ & \verb|\upsilon| \\
    $\gamma$      & \verb|\gamma|      & $\iota$     & \verb|\iota|     & 
        $\varpi$    & \verb|\varpi|   & $\phi$     & \verb|\phi| \\
    $\delta$      & \verb|\delta|      & $\kappa$    & \verb|\kappa|    & 
        $\rho$      & \verb|\rho|     & $\varphi$  & \verb|\varphi| \\
    $\epsilon$    & \verb|\epsilon|    & $\lambda$   & \verb|\lambda|   & 
        $\varrho$   & \verb|\varrho|  & $\chi$     & \verb|\chi| \\
    $\varepsilon$ & \verb|\varepsilon| & $\mu$       & \verb|\mu|       & 
        $\sigma$    & \verb|\sigma|   & $\psi$     & \verb|\psi| \\
    $\zeta$       & \verb|\zeta|       & $\nu$       & \verb|\nu|       & 
        $\varsigma$ & \verb|\varsigma|   & $\omega$   & \verb|\omega| \\
    $\eta$        & \verb|\eta|        & $\xi$       & \verb|\xi|       & 
        &                 &            & \\
    $\Gamma$      & \verb|\Gamma|      & $\Lambda$   & \verb|\Lambda|   & 
        $\Sigma$    & \verb|\Sigma|   & $\Psi$     & \verb|\Psi| \\
    $\Delta$      & \verb|\Delta|      & $\Xi$       & \verb|\Xi|       & 
        $\Upsilon$  & \verb|\Upsilon| & $\Omega$   & \verb|\Omega| \\
    $\Theta$      & \verb|\Theta|      & $\Pi$       & \verb|\Pi|       & 
        $\Phi$      & \verb|\Phi|     &            & \\
    \bottomrule
\end{tabular}
\end{table}

\subsection{指數、下標、根號}
指數或上標用~\verb|^|~表示,下標用~\verb|_|~表示,根號用~\verb|\sqrt|~表示。上下標如果多於一個字母或符號,需要用一對~\verb|{}|~括起來。
\begin{demo}
\[x_{ij}^2\quad \sqrt[2]{x}\]
\end{demo}

\subsection{分數}
分數用~\verb|\frac|~命令表示,它會自動調整字號,比如在行間公式中小一點,在獨立公式則大一點。\verb|\dfrac|~命令把分數的字號顯式設置為獨立公式中的大小,\verb|\tfrac|~命令則把字號設為行間公式中的大小。
\begin{demo}
$\frac{1}{2} \dfrac{1}{2}$
\[\frac{1}{2} \tfrac{1}{2}\]
\end{demo}

\subsection{運算符}
有些小的運算符(operator)例如~\verb|+ - * /|~等可以直接輸入,另一些則需要特殊命令。完整的數學符號參見~Scott Pakin~的《The Comprehensive \LaTeX~ Symbol List》\citep{Pakin_2008}。
\begin{code}
\[\pm \times \div \cdot \cap \cup \geq \leq \neq \approx \equiv\]
\end{code}

\begin{out}
\[\pm\quad \times\quad \div\quad \cdot\quad \cap\quad \cup\quad \geq\quad \leq\quad \neq\quad \approx\quad \equiv\]
\end{out}

和、積、極限、積分等大運算符用~\verb|\sum \prod \lim \int|~等表示。它們的上下標在行間公式中被壓縮,以適應行高。。

\begin{code}
$\sum_{i=1}^n i \prod_{i=1}^n \lim_{x\to0}x^2 \int_a^b x^2 dx\$
\[\sum_{i=1}^n i \prod_{i=1}^n \lim_{x\to0}x^2 \int_a^b x^2 dx\]
\end{code}

\begin{out}
$\sum_{i=1}^n i\quad \prod_{i=1}^n\ \lim_{x\to0}x^2\ \int_a^b x^2\mathrm{d}x$
\[\sum_{i=1}^n i\quad \prod_{i=1}^n\ \lim_{x\to0}x^2\ \int_a^b x^2\mathrm{d}x\]
\end{out}

多重積分如果用多個~\verb|\int|~來輸入的話,積分號間距過寬。正確的方法是用~\verb|\iint \iiint \iiiint \idotsint|~等命令輸入。從下例中可以看出兩種方法的差異。

\begin{out}
\[\iint\quad \iiint\quad \iiiint\quad \idotsint\]
\[\int\int\quad \int\int\int\quad \int\int\int\int\quad \int\dots\int\]
\end{out}

\subsection{分隔符}
各種括號用~\verb|() [] \{\} \langle\rangle|~等命令表示,注意花括號通常用來輸入命令和環境的參數,所以在數學公式中它們前面要加~\verb|\|。因為~\LaTeX~中的~\verb+|和\|+~的應用過於隨意,~amsamth~宏包推薦用~\verb|\lvert\rvert|~和~\verb|\lVert\rVert|~取而代之。

我們可以在上述分隔符前面加~\verb|\big \Big \bigg \Bigg|~等命令來調整大小。\LaTeX~原有的方法是在分隔符前面加~\verb|\left \right|~來自動調整大小,但是效果不佳,所以~amsmath~不推薦用這種方法。

\begin{out}
\[\Bigg(\bigg(\Big(\big((x+y)\big)\Big)\bigg)\Bigg)\quad 
\Bigg[\bigg[\Big[\big[[x+y]\big]\Big]\bigg]\Bigg]\quad 
\Bigg\{\bigg\{\Big\{\big\{\{x+y\}\big\}\Big\}\bigg\}\Bigg\}\]
\[\Bigg\langle\bigg\langle\Big\langle\big\langle\langle x+y \rangle\big\rangle\Big\rangle\bigg\rangle\Bigg\rangle\quad
\Bigg\lvert\bigg\lvert\Big\lvert\big\lvert\lvert x+y \rvert\big\rvert\Big\rvert\bigg\rvert\Bigg\rvert\quad
\Bigg\lVert\bigg\lVert\Big\lVert\big\lVert\lVert x+y \rVert\big\rVert\Big\rVert\bigg\rVert\Bigg\rVert\]
\end{out}

\subsection{箭頭}
\Fref{tab:arrow}~列出了部分箭頭的輸入方法。另外還有兩個命令生成的箭頭可以根據上下標自動調整長度。

\begin{table}[htbp]
\caption{箭頭}
\label{tab:arrow}
\centering
\begin{tabular}{llll}
    \toprule
    $\leftarrow$       & \verb|\leftarrow|      & $\longleftarrow$       & \verb|\longleftarrow| \\
    $\rightarrow$      & \verb|\rightarrow|     & $\longrightarrow$      & \verb|\longrightarrow| \\
    $\leftrightarrow$  & \verb|\leftrightarrow| & $\longleftrightarrow$  & \verb|\longleftrightarrow| \\
    $\Leftarrow$       & \verb|\Leftarrow|      & $\Longleftarrow$       & \verb|\Longleftarrow| \\
    $\Rightarrow$      & \verb|\Rightarrow|     & $\Longrightarrow$      & \verb|\Longrightarrow| \\
    $\Leftrightarrow$  & \verb|\Leftrightarrow| & $\Longleftrightarrow$  & \verb|\Longleftrightarrow| \\
    \bottomrule
\end{tabular}
\end{table}

\begin{demo}
\[\xleftarrow{x+y+z}\quad
\xrightarrow[x<y]{a*b*c}\]
\end{demo}

\subsection{標註}
\Fref{tab:accent}~列出一些短的注音符號(accent),\Fref{tab:notation}~則列出一些長的標註符號。

\begin{table}[htbp]
\caption{數學注音符號}
\label{tab:accent}
\centering
\begin{tabular}{llllllll}
    \toprule
    $\acute{x}$ & \verb|\acute{x}| & $\tilde{x}$   & \verb|\tilde{x}|   & $\mathring{x}$   & \verb|\mathring{x}|  \\
    $\grave{x}$ & \verb|\grave{x}| & $\breve{x}$ & \verb|\breve{x}| & $\dot{x}$   & \verb|\dot{x}|    \\
    $\bar{x}$  & \verb|\bar{x}|  & $\check{x}$ & \verb|\check{x}| & $\ddot{x}$  & \verb|\ddot{x}|  \\
    $\vec{x}$ & \verb|\vec{x}| & $\hat{x}$   & \verb|\hat{x}|   & $\dddot{x}$ & \verb|\dddot{x}| \\
    \bottomrule
\end{tabular}
\end{table}

\begin{table}[htbp]
\caption{長標註符號}
\label{tab:notation}
\centering
\begin{tabular}{llll}
    \toprule
    $\overline{xxx}$        & \verb|\overline{xxx}|        & $\overleftrightarrow{xxx}$  & \verb|\overleftrightarrow{xxx}| \\
    $\underline{xxx}$       & \verb|\underline{xxx}|       & $\underleftrightarrow{xxx}$ & \verb|\underleftrightarrow{xxx}| \\
    $\overleftarrow{xxx}$   & \verb|\overleftarrow{xxx}|   & $\overbrace{xxx}$           & \verb|\overbrace{xxx}| \\
    $\underleftarrow{xxx}$  & \verb|\underleftarrow{xxx}|  & $\underbrace{xxx}$          & \verb|\underbrace{xxx}| \\
    $\overrightarrow{xxx}$  & \verb|\overrightarrow{xxx}|  & $\widetilde{xxx}$           & \verb|\widetilde{xxx}| \\
    $\underrightarrow{xxx}$ & \verb|\underrightarrow{xxx}| & $\widehat{xxx}$             & \verb|\widehat{xxx}| \\
    \bottomrule
\end{tabular}
\end{table}

\subsection{省略號}
省略號用~\verb|\dots \cdots \vdots \ddots|~等命令表示,注意~\verb|\cdots|~和~\verb|\dots|~的差別。
\begin{out}
\[\dots\quad \cdots\quad \vdots\quad \ddots\]
\end{out}

\subsection{空白間距}
在數學模式中,我們可以用\Fref{tab:quad}中的命令生成不同的間距,注意負間距命令~\verb|\!|~可以用來減小間距。

\begin{table}[htbp]
\caption{空白間距}
\label{tab:quad}
\centering
\begin{tabular}{llll}
    \toprule
    \verb|\,| & 3/18 em & \verb|\quad|  & 1 em \\    
    \verb|\:| & 4/18 em & \verb|\qquad| & 2 em \\    
    \verb|\;| & 5/18 em & \verb|\!|     & -3/18 em \\
    \bottomrule
\end{tabular}
\end{table}

\section{矩陣和行列式}
數學模式下可以用~\verb|array|~環境來生成行列表。參數~\verb|{ccc}|~用於設置每列的對齊方式,\verb|l、c、r|~分別表示左中右;\verb|\\|~和~\verb|&|~用來分隔行和列。

\begin{demo}
\[\begin{array}{ccc}
x_1 & x_2 & \dots \\
x_3 & x_4 & \dots \\
\vdots & \vdots & \ddots \\
\end{array}\]
\end{demo}

\verb|amsmath|~有幾個類似的環境:\verb|pmatrix、bmatrix、Bmatrix、vmatrix|~和~
\verb|Vmatrix|,它們和~\verb|array|~的主要區別是會在表兩端加上~$()\; []\; \{\}\; ||\; \|\|$~等分隔符,其次這些環境沒有列對齊方式參數。

行間公式可以用~\verb|smallmatrix|~環境來生成排列緊密的小矩陣。

\section{多行公式}
有時一個公式太長一行放不下,或幾個公式需要寫成一組,這時我們就要用到~\verb|amsmath|~提供的幾個適合多行公式的環境。

\subsection{長公式}
對於多行不需要對齊的長公式,我們可以用~\verb|multiline|~環境。
\begin{demo}
\begin{multline}
x=a+b+c+\\
d+e+f+g
\end{multline}
\end{demo}

需要對齊的長公式可以用~\verb|split|~環境,它本身不能單獨使用,因此也稱作次環境,必須包含在~\verb|equation|~或其它數學環境內。\verb|split|~環境用~\verb|\\|~和~\verb|&|~來分行和設置對齊位置。
\begin{demo}
\[ \begin{split}
x=&a+b+c+\\
  &d+e+f+g
\end{split} \]
\end{demo}

\subsection{公式組}
不需要對齊的公式組用~\verb|gather|環境,需要對齊的用~\verb|align|。
\begin{demo}
\begin{gather}
a=b+c+d\\
x=y+z
\end{gather}
\end{demo}

\begin{demo}
\begin{align}
a&=b+c+d\\
x&=y+z
\end{align}
\end{demo}

\verb|multine、gather、align|~等環境都有帶~\verb|*|~的版本,它們不生成公式編號。

有多種條件的公式組用~\verb|cases|~次環境。
\begin{demo}
\[ y=\begin{cases}
-x & x<0\\
x & x\geq0
\end{cases} \]
\end{demo}

\section{定理和證明}
\LaTeX~提供了一個~\verb|\newtheorem|~命令來定義定理之類的環境,其語法如下。
\begin{code}
\newtheorem{環境名}[編號延續]{顯示名}[編號層次]
\end{code}

在下例中,我們定義了四個環境:定義、定理、引理和推論,它們都在一個~\verb|section|~內編號,而引理和推論會延續定理的編號。
\begin{code}
\newtheorem{defination}{定義}[section]
\newtheorem{theorem}{定理}[section]
\newtheorem{lemma}[theorem]{引理}
\newtheorem{corollary}[theorem]{推論}
\end{code}

\newtheorem{defination}{定義}[section]
\newtheorem{theorem}{定理}[section]
\newtheorem{lemma}[theorem]{引理}
\newtheorem{corollary}[theorem]{推論}

定義了上述環境之後,我們就可以像下面這樣使用它們。
\begin{demo}
\begin{defination}
Java是一種跨平台的編程語言。
\end{defination}
\end{demo}

\begin{demo}
\begin{theorem}
咖啡因會使人的大腦興奮。
\end{theorem}
\end{demo}

\begin{demo}
\begin{lemma}
茶和咖啡都會使人興奮。
\end{lemma}
\end{demo}

\begin{demo}
\begin{corollary}
晚上喝咖啡會導致失眠。
\end{corollary}
\end{demo}

\verb|proof|環境可以用來輸入下面這樣的證明,它會在證明結尾輸入一個~QED~符號\footnote{拉丁語~quod erat demonstrandum~的縮寫。}。

\begin{demo}
\begin{proof}[命題``物質無限可分''的證明]
一尺之棰,日取其半,萬世不竭。
\end{proof}
\end{demo}

\section{數學字體}
和文本模式類似,數學模式下可以用\Fref{tab:math_font}~中的命令選擇不同字體,其中有些字體需要加載~\verb|amsfonts|~宏包。

\begin{table}[htbp]
\caption{數學字體}
\label{tab:math_font}
\centering
\begin{tabular}{ccccccc}
    \toprule
    預設 & \verb|\mathbf| & \verb|\mathit| & \verb|\mathsf| & 
        \verb|\mathcal| & \verb|\mathbb| \\
    XNZRC & $\mathbf{XNZRC}$ & $\mathit{XNZRC}$ & $\mathsf{XNZRC}$ & 
        $\mathcal{XNZRC}$ & $\mathbb{XNZRC}$ \\
    \bottomrule
\end{tabular}
\end{table}

\bibliographystyle{unsrtnat}
\bibliography{reading}
\newpage


