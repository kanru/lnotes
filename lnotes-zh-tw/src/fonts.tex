\chapter{字體}

關於字體有三個重要概念:glyph、typeface、font。glyph~通常被翻譯為字形,也有翻譯為字體的;typeface~是一個書法和印刷領域的概念,它通常被翻譯為字體或書體;font~曾經和~typeface~混用,但現在一般用作電腦領域的概念,在中國大陸被翻譯為字體,在台灣被翻譯為字型。

上述翻譯的混亂令人十分無奈,包老師決定在本文中把它們分別翻譯為字形、字樣、字體,以正視聽。

字形是一個字符的具體圖形表現形式,一個字符可以有多個字形,比如漢字中的「{\CJKfamily{uming}強/强}」、「{\CJKfamily{uming}戶/户/戸}」;字樣是一組相同風格樣式的字形的集合,比如中文字樣有宋、仿、楷、黑、隸、篆等;一種字樣可以對應電腦上的幾種字體。

\section{字樣}
\label{sec:typeface}
拉丁字母的字樣主要有三大類:Serif(Roman)、Sans Serif~和~Mono-space~(Typewriter)。Serif~的筆畫邊緣部分有些裝飾,類似於中文的宋體、仿宋、楷體、魏書等。Sans Serif~的筆畫則是平滑的,類似於中文的黑體。Sans~這個詞來源於法語,就是「沒有」的意思。Monospace~則是等寬字樣。

每一類字樣都可以有加粗(bold)、斜體(italic)、傾斜(oblique)等修飾效果。Italic~通常對原字樣進行過重新設計,它修飾精細,多用於~Serif;~Oblique~也稱作~slanted,基本上是把正體傾斜,多用於~Sans Serif。通常~oblique~看起來比~italic~要寬一些。

\Fref{tab:typeface}列出了幾種常見的字樣。

\begin{table}[htbp]
\caption{常見字樣}
\label{tab:typeface}
\centering
\begin{tabular}{llll}
    \toprule
    作業系統 & Serif           & Sans Serif & Monospace \\
    \midrule
    Mac OS   & Times           & Helvetica  & Courier \\
    Windows  & Times New Roman & Arial      & Courier New \\
    \bottomrule
\end{tabular}
\end{table}

\section{字體格式}
\label{sec:font}

\subsection{點陣字體和向量字體}
電腦上用的字體(font)按數據格式可以分為三大類:點陣字體(bitmap)、輪廓(outline)字體和筆畫(stroke-based)字體。

點陣字體通過點陣來描述字形。早期的電腦受到容量和繪圖速度的限制,多採用點陣字體。點陣字體後來漸漸被輪廓字體所取代,但是很多小字號字體仍然使用它,因為這種情況下輪廓字體縮放太多會導致筆畫不清晰。

輪廓字體又稱作向量字體,它通過一組直線段和曲線來描述字形。輪廓字體易於通過數學函數進行縮放等變換,形成平滑的輪廓。輪廓字體的主要缺陷在於它所採用的貝茲曲線(Bézier curves)在光柵(raster)設備(比如顯示器和打印機)上不能精確渲染,因而需要額外的補償處理比如字體微調(font hinting)。但是隨著電腦硬件的發展,人們一般不在意它比點陣字體多出的處理時間。

筆畫字體其實也是輪廓字體,不過它描述的不是完整的字形,而是筆畫。它多用於東亞文字。

\subsection{常見字體}

常見的輪廓字體技術有:Type 1~和~Type 3、TrueType、OpenType、~\MF~等。

Adobe~的~Type 1~和~Type 3~基於~PS,它們採用三次貝茲曲線。Type 1~支持微調,它使用一個簡化的~PS~子集;Type 3~不支持微調,但它可以使用全部~PS~功能,因此既可以包含輪廓字體也可以包含點陣字體信息。
    
1991~年,Apple~發佈了~TrueType,它採用二次貝茲曲線。二次曲線處理起來比三次曲線快,但是需要更多的點來描述。所以從~TrueType~到~Type 1的轉換是無損的,反之是有損的。1994~年,Apple~著手研究~TrueType~的下一代技術:TrueType GX,它後來演變為Apple Advanced Typography(AAT)。

1996~年,微軟和~Adobe~聯合發佈了~OpenType。它比起~AAT~的優勢有:跨平台、開放和易於開發、支持更多的語言比如阿拉伯語。

早在~1984~年~Knuth~就發佈了~\MF,它與~TrueType~和~OpenType的區別是,不直接描述字形輪廓,而描述生成輪廓的筆的軌跡。筆的形狀可以是橢圓形或多邊形,尺寸縮放自如,字形邊緣也柔和一些。兩種字體可以用同一個~\MF~文件,當然還有不同的參數。\MF~技術如此先進,卻沒有流行開來。對此~Knuth~解釋道,要求一位設計字體的藝術家掌握~60~個參數太變態了,那是用來折磨數學家的。

Type 1~和~Type 3~把字體信息存儲在兩種文件裡:metrics~和~glyph~文件。metrics~文件有~AFM(Adobe font metrics)和~PFM(printer font metrics),glyph~文件有~PFA(printer font ASCII)和~PFB(printer font binary)。~\LaTeX~使用的~metrics~格式是~TFM(TeX Font Metrics)。

TrueType~的文件後綴是~.ttf,OpenType~的是~.ttf~和~.otf。\MF~雖然用向量圖形來定義字形,實際輸出的卻是一種點陣格式:PK~(packed raster)。

上述字體按技術的先進性,從高到低的排序為:OpenType、~TrueType、Type 1、Type 3、PK,我們應優先選用~OpenType~和~TrueType。

\subsection{合縱連橫}
Adobe~收取的~Type 1~專利許可費一度十分昂貴,窮人們只好用免費的~Type 3。為了打破這種壟斷,Apple~開發了~TrueType。1991~年~TrueType~發佈之後,Adobe隨即公開了~Type 1~的規範,~Type 1~字體從貴族墮落為平民,因而流行開來。

1980~年代中後期,Adobe~的大部分盈利來自於~PS~解釋器的許可費。面對這種壟斷局面,微軟和~Apple~聯合了起來。微軟把買來的~PS~解釋器~TrueImage~授權給~Apple,Apple~則把~TrueType~授權給微軟。

微軟得隆望蜀,又企圖獲得~AAT~的許可證,未遂。為了打破~Apple~的壟斷,微軟聯合~Adobe~在1996年發佈了~OpenType。Adobe~在2002~年末將其字體庫全面轉向~OpenType。

上面這幾齣精彩好戲充分展示了商場上的勾心鬥角、爾虞我詐,沒有永恆的夥伴,只有永恆的利益。但它同時也告訴我們,市場競爭中受益的還是廣大的消費者。

\section{字體應用}

PS~支持~Type 1~和~Type 3,而~PDF~除了這兩種還支持~TrueType~和~OpenType。\verb|latex|、DVI瀏覽器、各種~driver~分別採用不同的字體技術。

\subsection{DVI}
\verb|latex|~編譯~\LaTeX~源文件生成~DVI~時只需要\verb|.tfm|~文件,因為~DVI~並不包含字形信息,而只包含對字體的引用。DVI~瀏覽器顯示~DVI~時一般使用~PK,它在系統中查找相應的~\verb|.pk|~文件,若找不到就調用~\MF~在後台自動生成。

\subsection{dvips}
預設情況下,\verb|dvips|~也會查找~\verb|.pk|~,或調用~\MF~自動生成;然後把~PK~轉換成包含點陣字體的~Type 3,它的參數~\verb|-D|~可以用來控制該點陣字體的分辨率。用~\verb|ps2pdf|~處理含~Type 3~的~PS~時,輸出的自然是含~Type 3~的~PDF。

GSview~在低分辨率下可以很好地渲染~Type 3,Adobe Reader~或~Acrobat~卻不能,因為它們使用的~Adobe Type Manager~不支持包含完整~PS~的~Type 3。含~Type 3~的~PDF~看起來會有些模糊,所以應儘量避免使用。

\verb|dvips|~的另一個參數~\verb|-Ppdf|~把~Type 1~嵌入生成的~PS,這樣再~\verb|ps2pdf|~就能生成含~Type 1~的~PDF。

\verb|dvips|~不支持真正的(native)TrueType,用戶只能把~TrueType~先轉成~PK~或~Type 1,這樣繞了個彎效果總會打些折扣。

\verb|dvips|~的字體詳細使用方法可查閱其手冊\citep{Rokicki_2005}第~6~章,此處不贅述。

\subsection{dvipdfm(x)}
\verb|dvipdfm|~支持~PK~和~Type 1,它可以用一個~\verb|t1fonts.map|~文件建立~PK~文件和~Type 1~文件之間的映射,這樣生成的~PDF~用的就是~Type 1。\verb|dvipdfm|~也不支持真正的~TrueType。

\verb|dvipdfmx|~通過正確的設置可以使用真正的~TrueType,它對中日韓等東亞文字的支持也較好,所以它對我們來說是~Driver~的首選。

\section{TrueType~字體安裝配置}
CJK~自帶的~UTF-8~編碼字體~gbsn~和~gkai只包含~GB2312~字符集,而~CTeX~只提供~GBK~編碼字體,因此中文用戶通常需要自己安裝配置~UTF-8~編碼的~TrueType~字體。

在使用~TrueType~之前,用戶通常需要作以下準備工作:
\begin{enumerate}
    \item 用轉換程序~\verb|ttf2tfm|~生成~TFM。
    \item 配置字體定義文件~\verb|.fd|。
    \item 配置~\verb|ttf2pk|~,因為~DVI~瀏覽器和~\verb|dvips|~都會自動調用~\verb|ttf2pk|~來生成~PK。
    \item 配置~\verb|dvipdfmx|。
\end{enumerate}

\subsection{目錄和文件}
通常每個發行包都會參照~TDS~建立自己的目錄系統,把各種文件發在固定的位置。比如~MiKTeX~頂層目錄如下,在本節後面的示例中我們將使用這些目錄的縮寫。
\begin{code}
Install: D:\edit\MiKTeX 2.7
UserData: C:\Documents and Settings\Alpha\Local Settings\
    Application Data\MiKTeX\2.7
UserConfig: C:\Documents and Settings\Alpha\
    Application Data\MiKTeX\2.7
\end{code}

目錄多了有個缺點,文件不知道放在哪裡好。MiKTeX~中有的配置文件居然在四個目錄下各有一份,實在是令人髮指。幸好我們可以用下面的命令檢查配置文件的具體名字和路徑。

\begin{code}
initexmf --edit-config-file=ttf2pk
\end{code}

\subsection{ttf2tfm}
比如我們想把~\verb|SimSun18030.ttc|(18030~字符集的新宋體)轉換為~UTF8~編碼的字體文件,我們需要執行以下步驟。

\begin{enumerate}
    \item 把需要的~\verb|.ttf|~文件複製到~\verb|UserData/fonts/truetype/chinese/|。
    \item 用下面的命令生成\verb|.tfm|~和~\verb|.enc|~文件。
    \item 把~\verb|*.tfm|~複製到~\verb|UserData/fonts/tfm/chinese/utf8song/|。
    \item 把~\verb|*.enc|~複製到~\verb|UserData/fonts/enc/chinese/utf8song/|。
\end{enumerate}

\begin{code}
ttf2tfm SimSun18030.ttc -q -w utf8song@Unicode@
\end{code}

\subsection{字體定義文件}
字體定義文件將字體引用名和實際的字體文件聯繫起來,比如我們在~\verb|CJK|~環境中引用~\verb|usong|~時,系統將會找到並使用~\verb|utf8song*.tfm|。

\begin{code}
%UserData\tex\latex\CJK\UTF8\C70usong.fd
\ProvidesFile{c70usong.fd}
%character set: GB18030
%font encoding: Unicode
\DeclareFontFamily{C70}{usong}{\hyphenchar \font\m@ne}
\DeclareFontShape{C70}{usong}{m}{n}{<-> CJK * utf8song}{}
\DeclareFontShape{C70}{usong}{m}{it}{<-> CJK * utf8song}{}
\DeclareFontShape{C70}{usong}{bx}{n}{<-> CJKb * utf8song}{
    \CJKbold}
\endinput
\end{code}

\subsection{配置~\texttt{ttf2pk}}
MiKTeX~中~\verb|ttf2pk|~的配置文件是~\verb|ttf2pk.ini|~,在其它的發行包中可能是~\verb|ttf2pk.cfg|。

\verb|ttf2pk.ini|~中有一個~\verb|.map|~文件列表,後者定義了~TrueType~應該按編碼轉為~PK~等信息。

比如下面這個文件列表會讓~\verb|ttf2pk|~讀取~\verb|foo.map|~和~\verb|bar.map|。

\begin{code}
map foo.map
map bar.map
\end{code}

如果係統找不到~\verb|ttf2pk.ini|,它會預設使用~\verb|ttfonts.map|。
\begin{code}
%UserData\ttf2tfm\base\ttfonts.map
utf8song@Unicode@ SimSun18030.ttc
\end{code}

\subsection{配置~\texttt{dvipdfmx}}
配置~\verb|dvipdfmx|~是為了讓~PDF~正確地嵌入~TrueType,否則生成的文件中的內容不能複製、黏貼。

\begin{code}
%UserConfig\dvipdfm\config\dvipdfmx.cfg
f cid-x.map
\end{code}

\begin{code}
%UserData\dvipdfm\config\cid-x.map
utf8song@Unicode@ unicode SimSun18030.ttc
\end{code}

\bibliographystyle{unsrtnat}
\bibliography{reading}
\newpage


