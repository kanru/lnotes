\chapter{跋}

首先向一路披荊斬棘看到這裡的讀者表示祝賀,至少在精神上你已經成為一名合格的~\LaTeX{}er。從此你生是~\LaTeX~的人,死是~\LaTeX~的鬼。Once Black, never back。沒有堅持到這裡的同學自然已經重新投向「邪惡」的~MS Word,畢竟那裡點個按鈕就可以插入圖形,點個下拉框就可以選擇字體。

當然~\LaTeX{}er也有簡單的出路,就是只使用預設設置,儘量少用插圖;不必理會點陣、向量,也不必理會~Type 1、Type 3、TrueType、OpenType。因為內容高於形式,你把文章的版面、字體搞得再漂亮,它也不會因此成為《紅樓夢》;而《紅樓夢》即使是手抄本,也依然是不朽的名著。

包老師曾經以為~\LaTeX~和~Word~的關係就好像是《笑傲江湖》中華山的氣宗和劍宗,頭十年劍宗進步快,中間十年打個平手,再往後氣宗就遙遙領先。至於令狐沖的無招勝有招,風清揚的神龍見首不見尾又是另一重境界,普通人恐怕只能望其頸背。

費盡九牛二虎之力熬到本文殺青的時候,才發現從前的想法很傻很天真。讓我們揮一揮衣袖,不帶走一片雲,臥薪嘗膽忍辱負重,耐心等待~\XeTeX~和~Lua\TeX。

