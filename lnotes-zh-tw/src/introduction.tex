\chapter{簡介}

\begin{quotation}
滾滾長江東逝水,浪花淘盡英雄。是非成敗轉頭空。青山依舊在,幾度夕陽紅。

白髮漁樵江渚上,慣看秋月春風。一壺濁酒喜相逢。古今多少事,都付笑談中。
\begin{flushright}
---~楊慎《臨江仙》
\end{flushright}
\end{quotation}

\section{歷史回顧}

\LaTeX~是一種面向數學和其它科技文檔的電子排版系統。一般人們提到的~\LaTeX~是一個總稱,它包括~\TeX、\LaTeX、\AmS-\LaTeX~等\footnote{一般認為~\TeX~是一種引擎,\LaTeX~是一種格式,而~\AmS-\LaTeX~等是宏集。此處目的是簡介,故不展開討論。}。

\TeX~的開發始於~1977~年~5~月,Donald E. Knuth\footnote{斯坦福大學計算機系教授,已退休。}開發它的初衷是用於《The Art of Computer Programming》的排版。1962~年~Knuth~開始寫一本關於編譯器設計的書,原計劃是~12~章的單行本。不久~Knuth~覺得此書涉及的領域應該擴大,於是越寫越多,如滔滔江水連綿不絕,又如黃河氾濫一發不可收拾。1965~年完成的初稿居然有~3000~頁,全是手寫的!據出版商估計,這些手稿印刷出來需要~2000~頁,出書的計劃只好改為七卷,每卷一或兩章。1976~年~Knuth~改寫第二卷的第二版時,很鬱悶地發現第一卷的鉛版不見了,而當時電子排版剛剛興起,質量還差強人意。於是~Knuth~仰天長嘯:「我要扼住命運的咽喉」,決定自己開發一個全新的系統,這就是~\TeX。

1978~年~\TeX~第一版發佈後好評如潮,Knuth~趁熱打鐵在~1982~年發佈了第二版。人們現在使用的~\TeX~基本就是第二版,中間只有一些小的改進。1990~年~\TeX~ v3.0~發佈後,Knuth~宣佈除了修正~bug~外停止~\TeX~的開發,因為他要集中精力完成那本巨著的後幾卷\footnote{已出版的前三卷是:《Fundamental Algorithms》、《Seminumerical Algorithms》、《Sorting and Searching》;第四卷《Combinatorial Algorithms》和第五卷《Syntactic Algorithms》正在寫作中,預計~2015~年出版;第六卷《Theory of Context-free Languages》和第七卷《Compiler Techniques》尚未安排上工作日程。}。此後每發佈一個修正版,版本號就增加一位小數,使得它趨近於$\pi$(目前是~3.141592)。Knuth~希望將來他離世時,\TeX~的版本號永遠固定下來,從此人們不再改動他的代碼。他開發的另一個軟件~\MF~也作類似處理,它的版本號趨近於$e$,目前是~2.71828。

\TeX~是一種語言也是一個宏處理器,這使得它很好很強大,但是它同時又很繁瑣,讓人難以接近。因此~Knuth~提供了一個對~\TeX~進行了封裝的宏集~Plain \TeX,裡面有一些高級命令,有了它最終用戶就無須直接面對枯燥無味的~\TeX。

然而~Plain \TeX~還是不夠高級,所以~Leslie Lamport\footnote{現供職於微軟研究院。}在~80~年代初期開發了另一個基於~\TeX~的宏集~\LaTeX。1992~年~\LaTeX~ v2.09~發佈後,Lamport~退居二線,之後的開發活動由~Frank Mittelbach~領導的~The LaTeX Team~接管。此小組發佈的最後版本是~1994~年的~\LaTeXe,他們同時還在進行~\LaTeX~3~的開發,只是正式版看起來遙遙無期。

起初,美國數學學會(American Mathematical Society,AMS)看著\TeX~是好的,就派~Michael Spivak~寫了~\AmS-\TeX,這項基於~Plain \TeX~的開發活動進行了兩年(1983--1985)。後來與時俱進的~AMS~又看著~\LaTeX~是好的,就想轉移陣地,但是他們的字體遇到了麻煩。恰好~Mittelbach~和~Rainer Schöpf(後者也是~LaTeX Team~的成員)剛剛發佈了~New Font Selection Scheme for \LaTeX(NFSS),AMS~看著還不錯,就拜託他們把~AMSFonts加入~\LaTeX,繼而在~1989~年請他們開發~\AmS-\LaTeX。\AmS-\LaTeX~發佈於~1990~年,之後它被整合為~\AmS~宏包,像其它宏包一樣可以直接運行於~\LaTeX。

\section{優點和缺點}
當前的文字處理系統大致可以分為兩種:標記語言(Markup Language)式的,比如\LaTeX;所見即所得(WYSIWYG)式的,比如~MS Word\footnote{其實~Word~也有自己的標記語言域代碼(field code),只是一般用戶不瞭解。}。

一般而言,\LaTeX~相對於所見即所得系統有如下優點:
\begin{itemize}
    \item 高質量\ 它製作的版面看起來更專業,數學公式尤其賞心悅目。
    \item 結構化\ 它的文檔結構清晰。
    \item 批處理\ 它的源文件是文本文件,便於批處理,雖然解釋(parse)源文件可能很費勁。
    \item 跨平台\ 它幾乎可以運行於所有電腦硬件和作業系統平台。
    \item 免費\ 多數~\LaTeX~軟件都是免費的,雖然也有一些商業軟件。
\end{itemize}

相應地,\LaTeX~的工作流程、設計原則,資源的缺乏,以及開發人員的歷史局限性等種種原因也導致了一些缺陷:
\begin{itemize}
    \item 製作過程繁瑣,有時需要反覆編譯,不能直接或實時看到結果。
    \item 宏包魚龍混雜,水準參差不齊,風格不夠統一。
    \item 排版風格比較統一,但因而缺乏靈活性。
    \item 用戶支持不夠好,文檔不完善。
    \item 對國際語言和字體的支持很差。
\end{itemize}

拋開~MS Word~不談,即使跟同為標記語言的~HTML/Web~系統相比,\LaTeX~也有一些不足之處。比如~Web~瀏覽器對~HTML~內容的渲染(render)比~DVI~瀏覽器對~\LaTeX~內容的渲染要快上許多,基本上可以算是實時。雖然~HTML~內容可能沒有~LaTeX~那麼複雜,但是~DVI~畢竟是已經被~\LaTeX~編譯過的格式。

還有一點令人困惑的是,有一部分~\LaTeX~陣營的人士習慣於稱對方為「邪惡的」或「出賣靈魂的」,如果昂貴的微軟系統應當為人詬病,那麼更貴的蘋果係統為何卻被人追捧?

2000~年有記者在採訪~Lamport~時問:「為什麼當前沒有高質量的所見即所得排版系統?」他回答道:「門檻太高了,一個所見即所得系統要做到~\LaTeX~當前的水平,工作量之大不是單槍匹馬所能完成\footnote{\TeX/\LaTeX~也不單單是那幾個大腕兒完成的,他們背後還有眾多默默無聞的小人物,比如當年~Knuth~手下的大批學生。此所謂一將功成萬骨枯。}。微軟那樣的大公司可以做,但是市場太小了。我偶爾也會想加入「Dark Side」,讓微軟給我一組人馬來開發一個這樣的系統。」(包老師註:他果然於次年加入微軟。)

竊以為這兩大陣營其實是蘿蔔青菜的關係,與其抱殘守缺、互相攻訐,不如各取所需;甚至可以捐棄前嫌、取長補短,共建和諧社會。

\section{軟件準備}
\label{sec:latexsoft}

\LaTeX~是一個軟件系統,同時也是一套標準。遵照這些標準,實現了(implement)所要求功能的軟件集合被稱為發行版(distribution)。與此類似的例子有~Java~和~Linux,比如SUN、IBM、BEA~等公司都有自己的~Java~虛擬機(JVM),它們都被稱作~Java~的實現;而Linux~有~Red Hat/~Fedora、Ubuntu、SuSE~等眾多的發行版。

\begin{table}[htbp]
\caption{\LaTeX~發行版與編輯器}
\label{tab:latexsoft}
\centering
\begin{tabular}{lll}
    \toprule
    作業系統 & 發行版 & 編輯器 \\
    \midrule
    Windows & \href{http://www.miktex.org/}{MikTeX} & \href{http://www.toolscenter.org/}{TeXnicCenter}、\href{http://www.winedt.com/}{WinEdt} \\
    Unix/Linux & \href{http://www.tug.org/texlive/}{TeX Live} & \href{http://www.gnu.org/software/emacs/emacs.html}{Emacs}、\href{http://vim.sourceforge.net/}{vim}、\href{http://kile.sourceforge.net/}{Kile} \\
    Mac OS & \href{http://www.tug.org/mactex/}{MacTeX} & \href{http://www.uoregon.edu/~koch/texshop/}{TeXShop} \\
    \bottomrule
\end{tabular}
\end{table}

\LaTeX~發行版只提供了一個~\LaTeX~後台處理機制,用戶還需要一個前台編輯器來編輯它的源文件。常用的~\LaTeX~發行版和編輯器見\fref{tab:latexsoft}。在使用~\LaTeX~的過程中可能還需要其它一些軟件,將在後面相關章節中分別介紹。

\section{學習方法}
\begin{quotation}
在科學上沒有平坦的大道,只有那些不畏勞苦沿著陡峭山路攀登的人,才有希望達到光輝的頂點。
\begin{flushright}
---~卡爾‧馬克思
\end{flushright}
\end{quotation}

\begin{quotation}
無他,唯手熟爾。
\begin{flushright}
---~賣油翁
\end{flushright}
\end{quotation}

\begin{quotation}
用心。
\begin{flushright}
---~斯蒂芬‧周
\end{flushright}
\end{quotation}

限於篇幅和水平,本文只能提供一個概覽外加一些八卦。比較嚴謹的入門資料有~Tobias Oetiker~的《A (Not So) Short Introduction to \LaTeXe》\citep{Oetiker_2008}(簡稱lshort);若想對~\LaTeX~有更深入全面的瞭解,可以拜讀~Mittelbach~的《The \LaTeX~ Companion》\citep{Mittelbach_2004}。

中文資料可參考李果正的《大家來學~\LaTeX》\citep{Lee_2004},lshort~有吳凌雲等人翻譯的中文版本\footnote{此譯本首發於~CTeX~論壇,但是需要註冊才能看見鏈接,所以請讀者自行搜索。}。

\href{http://www.ctan.org/}{Comprehensive TeX Archive Network}(CTAN)和~\href{http://www.tug.org/}{TeX Users Group}~(TUG)提供了權威、豐富的資源。

\href{http://www.text.ac.uk/}{英國TUG}~和~\href{http://www.ctex.org}{CTeX}~分別提供了常見問題集(FAQ)\citep{UKTUG_FAQ,CTeX_FAQ},一般問題多會在這裡找到答案。

中文~\TeX~論壇有\href{http://www.smth.org/bbsdoc.php?board=TeX}{水木清華~BBS TeX~版}、\href{http://bbs.ctex.org/}{CTeX~論壇}。

\bibliographystyle{unsrtnat}
\bibliography{reading}
\newpage

