\chapter{雜項}

\section{超鏈接}
\label{sec:hyperlink}

\verb|hyperref|~宏包\citep{Rahtz_2006}提供了一些超鏈接功能。它給文檔內部的交叉引用和參考文獻自動加上了超鏈接,還提供了幾個命令。

\verb|\hyperref|~命令對已經定義的label進行簡單包裝,加上文字描述。

\begin{code}
\usepackage{hyperref}
...
\label{sec:hyperlink}
...
例如\ref{sec:hyperlink}是編號形式的鏈接,而\hyperref[sec:hyperlink]{這個鏈接}是文字形式的鏈接,都指向本節開始。
\end{code}

\begin{out}
例如\ref{sec:hyperlink}是編號形式的超鏈接,而\hyperref[sec:hyperlink]{這個鏈接}則是文字形式,都指向本節開始。
\end{out}

\verb|\url|~和~\verb|\href|~命令可以用來定義外部鏈接,後者有文字描述。

\begin{code}
\url{http://www.dralpha.com/}
\href{http://www.dralpha.com/}{包老師的主頁}
\end{code}

\begin{out}
\url{http://www.dralpha.com/}

\href{http://www.dralpha.com/}{包老師的主頁}
\end{out}

%hyperref選項

\section{長文檔}
當文檔很長時,我們可以把它分為多個文件,然後在主控文檔的正文中引用它們。注意~\verb|\include|~命令會新起一頁,如果不想要新頁可以改用~\verb|\input|~命令。
\begin{code}
%master.tex
\begin{document}
\include{chapter1.tex}
\include{chapter2.tex}
...
\end{document}
\end{code}

當文檔很長時,編譯一遍也會很花時間,我們可以用~\verb|syntonly|~宏包。這樣編譯時就只檢查語法,而不生成結果文件。
\begin{code}
\usepackage{syntonly}
...
\syntaxonly
\end{code}

\section{參考文獻}
在文檔中,我們經常要引用參考文獻(bibliography)。\LaTeX~提供的~\verb|thebibliography|~環境和~\verb|\bibtem|~命令可以用來定義參考文獻條目及其列表顯示格式,\verb|cite|~命令用來在正文中引用參考文獻條目。這種方法把內容和格式混在一起,用戶需要為每個條目設置格式,很繁瑣且易出錯。

\subsection{BibTeX}
1985年,~Oren Patashnik\footnote{Wiki~上說他是~Knuth~的學生,我發現他不在~Knuth~的博士生列表上,而在姚期智的博士生列表上,也許他是~Knuth~的碩士生。}和~Lamport~開發了~\BibTeX\citep{Patashnik_1988},其詳細使用方法請參閱~Nicolas Markey~的《Tame the BeaST: The B to X of BibTeX》\citep{Markey_2005}

\BibTeX~把參考文獻的數據放在一個~\verb|.bib|~文件中,顯示格式放在~\verb|.bst|~文件中。普通用戶一般不需要改動~\verb|.bst|,只須維護~\verb|.bib|~數據庫。

一個~\verb|.bib|~文件可以包含多個參考文獻條目(entry),每個條目有類型、關鍵字,以及題目、作者、年份等字段。常用條目類型有~article、~book、conference、manual、misc、techreport~等。每種類型都有一些自己的規定字段和可選字段,字段之間用逗號分開。數據庫中每個條目的關鍵字要保持唯一,因為引用時要用到它們。

下例顯示了一個條目,它的類型是~\verb|manual|,關鍵字是~\verb|Markey_2005|。~\verb|.bib|~文件可以用普通文本編輯器來編輯,也可以用專門的文獻管理軟件來提高效率。包老師推薦~\href{http://jabref.sourceforge.net/}{JabRef}。

\begin{code}
@MANUAL{Markey_2005,
  title = {Tame the BeaST: The B to X of BibTeX},
  author = {Nicolas Markey},
  year = {2005},
  url = {http://www.ctan.org/tex-archive/info/bibtex/
    tamethebeast/}
}
\end{code}

有了數據庫,我們可以像下面這樣引用一個條目。
\begin{demo}
請參閱\cite{Markey_2005}。
\end{demo}

前文中我們提到含有交叉引用的文檔需要編譯兩遍。含有參考文獻的文檔更麻煩,它需要依次執行~\verb|latex、bibtex、latex、latex|~等四次編譯操作。

\begin{enumerate}
    \item 第一遍~\verb|latex|~只把條目的關鍵字寫到中間文件~\verb|.aux|~中去。
    \item \verb|bibtex|~根據\verb|.aux、.bib、.bst|~生成一個~\verb|.bbl|~文件,即參考文獻列表。它的內容就是~\verb|thebibliography|~環境和一些~\verb|\bibtem|~命令。
    \item 第二遍~\verb|latex|~把交叉引用寫到~\verb|.aux|~中去。
    \item 第三遍~\verb|latex|~則在正文中正確地顯示引用。
\end{enumerate}

\begin{figure}[htbp]
\centering
\begin{tikzpicture}
    \node[box] (tex) {.tex};
    \node[box, right=4 of tex] (aux) {.aux};
    \node[box, right=6 of aux] (bbl) {.bbl};
    \node[box, above=1.5 of aux] (bib) {.bib};
    \node[box, below=1.5 of aux] (bst) {.bst};
    \path (tex) edge [arrow] node[auto] {latex} (aux)
        (aux) edge [arrow] node[auto] {bibtex} (bbl)
        (bib.east) edge [rloop] (bst);
\end{tikzpicture}
\caption{\BibTeX~的編譯}
\label{fig:bibtex}
\end{figure}

注意在長文檔中使用參考文獻時,應該用~\verb|latex|~編譯主控文檔,而用~\verb|bibtex|~編譯子文檔。

\begin{code}
latex master(.tex)
bibtex chapter1(.tex)
latex master(.tex)
latex master(.tex)
\end{code}

\subsection{natbib}
參考文獻的引用通常有兩種樣式:作者-年份和數字。\LaTeX~本身只支持數字樣式,而~\verb|natbib|~宏包\citep{Daly_2007}則同時支持這兩種樣式。

使用~\verb|natbib|~宏包時,我們首先要引用宏包;其次設置文獻列表樣式和引用樣式,每種列表樣式都有自己的預設引用樣式,所以後者可選;然後指定參考文獻數據庫。
\begin{code}
\usepackage{natbib}
...
\begin{document}
\bibliographystyle{plainnat}
\setcitestyle{square,aysep={},yysep={;}}
\bibliography{mybib.bib}
...
\end{document}
\end{code}

\verb|natbib|~提供了三種列表樣式:plainnat、abbrvnat、unsrtnat。前兩種都是作者-年份樣式,文獻列表按作者-年份排序,後者會使用一些縮寫(比如作者的~first name);unsrtnat~是數字樣式,文獻列表按引用順序排序。

\verb|\setcitestyle|~命令可以用來改變引用樣式的設置,其選項見~\Fref{tab:citestyle}。

\begin{table}[htbp]
\caption{參考文獻引用樣式選項}
\label{tab:citestyle}
\centering
\begin{tabular}{ll}
    \toprule
    引用模式            & authoryear、numbers、super \\
    括號                & round、square、open={char},close={char} \\
    引用條目分隔符      & 分號、逗號、citesep={char} \\
    作者年份分隔符      & aysep={char} \\
    共同作者年份分隔符  & yysep={char} \\
    註解分隔符          & notesep={text} \\
    \bottomrule
\end{tabular}
\end{table}

注意在長文檔中,每個含參考文獻的子文檔都需要分別設置列表樣式,並指定數據庫。

\verb|natbib|~提供了多種引用命令,其中最基本的是~\verb|\citet|~和~\verb|\citep|~,它們在不同引用模式下效果不同。一般不推薦使用~\LaTeX~本身提供的~\verb|\cite|,因為它在作者-年份模式下和~\verb|\citet|~一樣,在數字模式下和~\verb|\citep|~一樣。

作者-年份模式下引用命令的效果如下。
\setcitestyle{authoryear}
\begin{demo}
參閱\cite{Daly_2007}\\
參閱\citet{Daly_2007}\\
參閱\citep{Daly_2007}
\end{demo}

數字模式下引用命令的效果如下。
\setcitestyle{numbers}
\begin{demo}
參閱\cite{Daly_2007}\\
參閱\citet{Daly_2007}\\
參閱\citep{Daly_2007}
\end{demo}

上標模式下引用命令的效果如下。
\setcitestyle{super}
\begin{demo}
參閱\cite{Daly_2007}\\
參閱\citet{Daly_2007}\\
參閱\citep{Daly_2007}
\end{demo}

另外還有一些引用命令,如~\verb|\citetext、\citenum、\citeauthor|、~\verb|\citeyear|~等,此處不贅述。

\section{索引}
\verb|makeidx|~宏包提供了索引功能。應用它時,我們首先需要在文檔序言部分引用宏包,並使用~\verb|makeindex|~命令;其次在正文中需要索引的地方定義索引,注意索引關鍵字在全文中須保持唯一;最後在合適的地方(一般是文檔末尾)打印索引。

\begin{code}
\usepackage{makeidx}
\makeindex
...
\begin{document}
\index{索引關鍵字}
...
\printindex
\end{document}
\end{code}

當編譯含索引的文檔時,用戶需要執行~\verb|latex、makeindex、latex|~等三次編譯操作。

\begin{enumerate}
    \item 第一遍~\verb|latex|~把索引條目寫到一個~\verb|.idx|~文件中去。
    \item \verb|makeindex|~把~\verb|.idx|~排序後寫到一個~\verb|.ind|~文件中去。
    \item 第二遍~\verb|latex|~在~\verb|\printindex|~命令的地方引用~\verb|.ind|~的內容,生成正確的DVI。
\end{enumerate}

\begin{figure}[htbp]
\centering
\begin{tikzpicture}
    \node[box] (tex) {.tex};
    \node[box, right=4 of tex] (idx) {.idx};
    \node[box, right=7 of idx] (ind) {.ind};
    \node[box, right=6 of ind] (dvi) {.dvi};
    \node[box, above=1.5 of ind] (tex1) {.tex};
    \node[right=1 of ind] (point) {};
    \path (tex) edge [arrow] node[auto] {latex} (idx)
        (idx) edge [arrow] node[auto] {makeindex} (ind)
        (ind) edge [arrow] node[auto] {latex} (dvi)
        (tex1.east) edge [rloop] (point);
\end{tikzpicture}
\caption{索引的編譯}
\label{fig:makeidx}
\end{figure}

\section{頁面佈局}
在~\LaTeX~中用戶可以通過~\verb|\pagestyle|~和~\verb|\pagenumbering|~命令來設置頁眉(header)、頁腳(footer)的樣式和內容。頁面樣式有以下四種。

\begin{table}[htbp]
\caption{\LaTeX~頁面樣式}
\centering
\begin{tabular}{ll}
    \toprule
    \texttt{empty} & 頁眉、頁腳空白 \\
    \texttt{plain} & 頁眉空白,頁腳含居中頁碼 \\
    \texttt{headings} & 頁腳空白,頁眉含章節名和頁碼 \\
    \texttt{myheadings} & 頁腳空白,頁眉含頁碼和用戶自定義信息 \\
    \bottomrule
\end{tabular}
\end{table}

\verb|fancyhdr|\citep{Oostrum_2004}宏包提供了更靈活的控制。我們可以用以下代碼定製頁眉、頁腳的內容,以及頁眉下方、頁腳上方的橫線。

\begin{code}
\usepackage{fancyhdr}
...
\pagestyle{fancy} %fancyhdr宏包新增的頁面風格
\lhead{左擎蒼}
\chead{三個代表}
\rhead{右牽黃}
\lfoot{左青龍}
\cfoot{八榮八恥}
\rfoot{右白虎}
\renewcommand{\headrulewidth}{0.4pt}
\renewcommand{\footrulewidth}{0.4pt}
\end{code}

\ \\
\begin{tikzpicture}
\draw (0,0) rectangle (36,10);
\node at(2,9) {左擎蒼};
\node at(18,9) {三個代表};
\node at(34,9) {右牽黃};
\draw (.5,8)--(35.5,8);
\node at(18,5) {和諧社會};
\draw (.5,2)--(35.5,2);
\node at(2,1) {左青龍};
\node at(18,1) {八榮八恥};
\node at(34,1) {右白虎};
\end{tikzpicture}

用戶可以在頁眉、頁腳中使用一些~\LaTeX~變量,比如分別代表頁碼和章節編號的~\verb|\thepage、\thechapter、\thesection|;代表章節起始單詞(Chapter、Section等)的~\verb|\chaptername、\sectionname|~等。

這些變量組合起來可以構成復合標記~\verb|\leftmark|~和~\verb|\rightmark|~。當文檔奇偶頁面佈局不同時,我們可以使用以下方法為奇偶頁分別設置頁眉、頁腳。\verb|fancyhdr|~宏包會自動把每章起始頁的樣式設為~\verb|plain|,若想去掉頁腳中間的頁碼,可以重定義~\verb|plain|~樣式。

\begin{code}
\pagestyle{fancy}
\fancyhf{}                  %清空頁眉頁腳
\fancyhead[LE,RO]{\thepage} %偶數頁左,奇數頁右
\fancyhead[RE]{\leftmark}   %偶數頁右
\fancyhead[LO]{\rightmark}  %奇數頁左
\fancypagestyle{plain}{     %重定義plain頁面樣式
    \fancyhf{}
    \renewcommand{\headrulewidth}{0pt}
}
\end{code}

\ \\
\begin{tikzpicture}
\draw (0,0) rectangle (36,10);
\node at(2.5,9) {3.2 節名};
\node at(35,9) {17};
\draw (.5,8)--(35.5,8);
\node at(18,5) {奇數頁};
\end{tikzpicture}

\ \\
\begin{tikzpicture}
\draw (0,0) rectangle (36,10);
\node at(1,9) {18};
\node at(32,9) {Chapter 3 章名};
\draw (.5,8)--(35.5,8);
\node at(18,5) {偶數頁};
\end{tikzpicture}

Lamport當初設計~\LaTeX~時把頁面佈局變量的定義方式搞得比較晦澀,用戶在重定義~\verb|\leftmark|~和~\verb|\rightmark|~時,不能直接用~\verb|\renewcommand|~的方法,而要用另外兩個命令。

\begin{code}
\markboth{main-mark}{sub-mark}
\markright{sub-mark}
\end{code}

\verb|\leftmark|~即~main-mark,是一種高層次標記,在~article~文檔類中它包含~section~的信息,在~report~和~book~則包含~chapter~的信息;\verb|rightmark|~則是一種低層次標記,在~article~中包含~subsection~信息,在~report~和~book~中包含~section~信息。

比如在~\verb|book|~文檔類中,章節標記是通過下面的方法定義的,其中的~\verb|#1|~指的是章節的名字。

\begin{code}
\renewcommand\chaptermark[1]{\markboth{\chaptername \thechapter. 
    #1}{}}
\renewcommand\sectionmark[1]{\markright{\thesection. #1}}
\end{code}

%\section{演示文檔}

\bibliographystyle{unsrtnat}
\bibliography{reading}
\newpage


